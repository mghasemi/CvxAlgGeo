\documentclass{amsart}

\linespread{1.3}

\usepackage{amsmath, amssymb, mathrsfs, amsthm, MnSymbol, xifthen, verbatim, color, mathrsfs, graphicx}%, etoolbox}
\usepackage[arrow, matrix]{xy}

\definecolor{DarkBlue}{rgb}{0,0.2,0.6}
\definecolor{PinkPurple}{rgb}{0.8,0.3,0.3}

\usepackage[pdftex,
            pdfauthor={Mehdi Ghasemi},
            pdftitle={CvxAlgGeo Manual},
            pdfsubject={Convex Algebraic Geometry},
            pdfkeywords={Sums of Squares, Optimization, Global, Constrained},
            pdfproducer={Latex with hyperref},
            pdfcreator={PDFLaTeX},
            linkcolor=PinkPurple,
            citecolor=blue,
            urlcolor=DarkBlue,
            colorlinks=true]{hyperref}

%%% Italic Environments %%%
\newtheorem{thm}{Theorem}[section]
\newtheorem{lemma}[thm]{Lemma}
\newtheorem{prop}[thm]{Proposition}
\newtheorem{crl}[thm]{Corollary}

%%% Plain Environments %%%
\theoremstyle{definition}
\newtheorem{dfn}[thm]{Definition}
\newtheorem{exm}[thm]{Example}
\newtheorem{qus}[thm]{Question}
\newtheorem{rem}[thm]{Remark}

%%% Sets %%%
\newcommand{\reals}{\mathbb{R}}
\newcommand{\naturals}{\mathbb{N}}
\newcommand{\integers}{\mathbb{Z}}
\newcommand{\cplx}{\mathbb{C}}
\newcommand{\rationals}{\mathbb{Q}}
\newcommand{\rx}{\sgr{\mathbb{R}}{\ux}}
\newcommand{\sos}{\sum\mathbb{R}[\underline{X}]^2}
\newcommand{\rfps}{\fps{\mathbb{R}}{\underline{X}}}
\newcommand{\sage}{\href{http://www.sagemath.org}{\textsc{Sage}}}
\newcommand{\cvxalggeo}{\href{https://github.com/mghasemi/CvxAlgGeo}{\textsc{CvxAlgGeo}}}

%%% Special Symbols %%%
\newcommand{\tvs}{\mathcal{TVS}}
\newcommand{\spr}{\textrm{Sper}}
\newcommand{\sgn}{\textrm{Sgn}}
\newcommand{\ux}{\underline{X}}
\newcommand{\ur}{\underline{r}}
\newcommand{\la}{\langle}
\newcommand{\ra}{\rangle}
\newcommand{\supp}{\text{supp}}
\newcommand{\spn}{\text{Span}}
\newcommand{\etal}{\textit{et al. }}

%%% One Parameter %%%
\newcommand{\ringsos}[1]{\sum #1^2}
\newcommand{\K}[1]{\mathcal{K}_{#1}}
\newcommand{\pos}[1]{\mbox{Psd}(#1)}
\newcommand{\etop}[1]{\mathcal{E}_{#1}}
\newcommand{\V}[1]{\mathcal{X}_{#1}}
\newcommand{\Z}[1]{\mathcal{Z}(#1)}
\newcommand{\T}[1]{\mathcal{T}_{#1}}
%\newcommand{\M}[1]{\sigma(#1)}
\newcommand{\M}[1]{\mathfrak{sp}(#1)}
\newcommand{\bdd}[1]{\ell^{\infty}(#1)}
%\newcommand{\M}[1]{\mathfrak{M}(#1)}
\newcommand{\rgp}[1]{#1_{\textrm{gp}}}
\newcommand{\rsos}[1]{#1_{\textrm{sos}}}

%%% Two Parameters %%%
\newcommand{\sgr}[2]{#1[#2]}
\newcommand{\ringsop}[2]{\sum #1^{#2}}
\newcommand{\sgrsos}[2]{\sum #1[#2]^2}
\newcommand{\fps}[2]{#1\lsem #2\rsem}
\newcommand{\qm}[2]{\mbox{QM}(#1; #2)}
\newcommand{\Hom}[2]{\mbox{Hom}(#1,#2)}
\newcommand{\norm}[2]{\|\ifthenelse{\isempty{#2}}{\cdot}{#2}\|_{#1}}
\newcommand{\cl}[2]{\overline{#2}^{\ifthenelse{\isempty{#1}}{}{#1}}}
\newcommand{\Cnt}[2]{\mathrm{C}_{\ifthenelse{\isempty{#1}}{}{#1}}(#2)}
\newcommand{\Psd}[2]{\mbox{Psd}_{\ifthenelse{\isempty{#1}}{}{#1}}(#2)}
\newcommand{\crgp}[2]{#1_{\textrm{gp},#2}}

%%% Three Parameters %%%
\newcommand{\map}[3]{#1:#2\longrightarrow #3}

\begin{document}

\title{CvxAlgGeo user manual}
%
\author[M. Ghasemi]{Mehdi Ghasemi}
%
\address{Department of Mathematics and Statistics,\newline\indent
University of Saskatchewan,\newline\indent
Saskatoon, SK. S7N 5E6, Canada}
%
\email{mehdi.ghasemi@usask.ca}
%
%
\keywords{sums of squares, semidefinite programming, geometric programming, optimization}

%\subjclass[2010]{Primary 13J30, 14P99, 44A60; Secondary 43A35, 46B99, 44A60.}

\date{\today}

%\begin{abstract}
%\end{abstract}

\maketitle

%\tableofcontents

%%%%%%%%%%%%%%%%%%%%%%%%%%%%%%%%%%%%%%%%%%%%%%%%%%%%%%%%%%%%%%%%%%%%%
\section{What is \cvxalggeo?}

\cvxalggeo~ is a package written for \sage~ and published under GNU(GPL) to implement a fast method to compute a lower bound for multivariate 
polynomials over a basic closed semialgebraic set. It is mainly based on methods introduced in \cite{lbgp}, \cite{lbgpLasserre} and \cite{genlbgp}. 
Moreover, to provide computational tools for usual calculation based on semidefinite programming \cvxalggeo~ implements some functionalities 
available on \textsc{SosTools} and \textsc{GloptiPoly} for \textsc{Matlab} \cite{gloptipoly, sostools}.

At this stage, \cvxalggeo~ contains implementations of two different methods for polynomial optimization:

\subsection{By Geometric Programming:}
Let $f=\sum_{\alpha}f_{\alpha}\ux^{\alpha}$ be a polynomials on $X_1,\dots, X_n$ of degree $d$, where $\ux^{\alpha}$ is short for 
$X_1^{\alpha_1}\cdots X_n^{\alpha_n}$, and $\alpha=(\alpha_1,\dots,\alpha_n)\in\naturals^n$. 
The degree of a monomial $\ux^{\alpha}$ is denoted by $|\alpha|$ that is defined by $|\alpha|=\alpha_1+\cdots+\alpha_n$. 
Denote $f_{\alpha}$ where $\alpha=2d(\delta_{1i},\dots,\delta_{ni})$ by $f_{2d,i}$ and $f_{(0,\dots,0)}=f(\underline{0})$ by $f_0$ for simplicity.
Degree of a polynomial $f$ ($\deg f$) is the maximum degree of its monomials.
We also denote the set of those exponents $\alpha$ for which the monomial $f_{\alpha}\ux^{\alpha}$ is not a square by $\Delta_f$, i.e.,
\[
    \Delta_f:=\{\alpha\in\naturals^n ~:~ f_{\alpha}<0\textrm{ or }\alpha_i\textrm{ is odd for some }1\leq i\leq n\}.
\]
Let $\textbf{g}=(g_1,\dots,g_m)$ be a tuple of polynomials and $K_{\textbf{g}}=\{x\in\reals^n : g_j(x)\ge 0, j=1,\dots,m\}$. Denote the infimum 
of $f$ over $K_{\textbf{g}}$ by $f_{*,\textbf{g}}$, $\Delta=\Delta_f\cup\Delta_{-g_1}\cup\dots\cup\Delta_{-g_m}$ and $A=(a_{kl})$ be a 
$(m+1)\times(m+1)$ matrix such that $a_{0l}=1$ if $l=1$ and $0$ otherwise. Define $h_k=\sum_{j=0}^m a_{jk}g_j$, $k=0,\dots,m$ where $g_0=-f$ and
$H(\mu)=-\sum_{j=0}^m \mu h_j$. We can Decompose $H$ as $H^+-H^-$ where all the coefficients of $H^+$ and $H^-$ are nonnegative.
If $\rho$ is an optimum value for the following \textit{geometric program}, then $f_{gp,\textbf{g}}=-h_0(0)-\rho$ is a lower bound for the global 
minimum of $f$ on $K_{\textbf{g}}$.
\begin{equation}\label{GP}
\left\{\begin{array}{rl}
&Minimize \ \sum\limits_{j=1}^m \mu_jh_j(0)^++ \sum\limits_{\alpha\in\Delta^{<d}}(d-\vert\alpha\vert)\left[
\left(\frac{w_\alpha}{d}\right)^{d}\,\left(\frac{\alpha}{\bold{z}_\alpha}\right)^\alpha
\right]^{1/(d-\vert\alpha\vert)}\\
&\\
&\mbox{s.t.}\ \sum\limits_{\alpha\in\Delta}z_{\alpha,i} \ \le \ H(\mu)_{d,i}\,,\quad i=1,\ldots,n\\
&\\
&\left(\frac{\bold{z}_\alpha}{\alpha}\right)^\alpha \,\ge \,\left(\frac{w_\alpha}{d}\right)^{d},\quad\alpha\in\Delta^{=d} \\
&\\
&w_{\alpha}\ge \max\{ H(\mu)_{\alpha}^+, H(\mu)_{\alpha}^-\}, \quad\alpha \in \Delta \\
&\\
&\sum\limits_{k=0}^m a_{jk}\mu_k \ge 0, \quad j=1,\dots,m
\end{array}\right.
\end{equation}
Here $h_j^+(0)=\max\{h_j(0),0\}$ \cite[Theorem 4.1]{genlbgp}.

In \cite{lbgpLasserre}, \cite{lbgp} and \cite{genlbgp} the advantages and disadvantages of solving \eqref{GP} instead of solving the corresponding 
semidefinite program is discussed.

\subsection{By Semidefinite Programming:}
It is  well known that a feasible value for the semidefinite program \eqref{CSOS} is a lower bound for the polynomial $f$ on a semialgebraic set $K=K_S$, where 
$S=\{g_1,\dots,g_m\}$ and $K_S=\{x\in\reals^n:g_i(x)\ge0, i=1,\dots,m\}$.
\begin{equation}\label{CSOS}
\textrm{Prg}_k:\left\lbrace\begin{array}{lll}
	\min\limits_y & \sum_{\alpha}f_{\alpha}y_{\alpha} & \\
	\textrm{subject to} & M_k(y)\succeq0 & \\
	 & M_{k-\deg g_i}(g_iy)\succeq0 & i=1,\dots,m
\end{array}\right.
\end{equation}
for $k\ge\max\{\deg f,\deg g_1,\dots,\deg g_m\}$. All the notations are as described in \cite{LasserreSDP}. Current implementation is based on the description 
given on \cite[Chapter 4 and 5]{LasserreMPP} and \cite{Laurent}.
%%%%%%%%%%%%%%%%%%%%%%%%%%%%%%%%%%%%%%%%%%%%%%%%%%%%%%%%%%%%%%%%%%%%%
\section{\cvxalggeo}

\cvxalggeo~ implements few tools for computations over real polynomial rings in Python for \sage~ using 
\href{http://abel.ee.ucla.edu/cvxopt/}{\textsc{CvxOpt}} and \href{https://projects.coin-or.org/Csdp/}{CSDP}.
The current version consists of three main modules.
\begin{enumerate}
	\item{\texttt{geometric}}
	\item{\texttt{semidefinite}}
    \item{\texttt{invariant}}
\end{enumerate}
In this section we describe how to use these programs.
	
%%%%%%%%%%%%%%%%%%%%%%%%%%%%%%%%%%%%%
\subsection{geometric.GPTools(Prog, Rng [, H=A, Settings])}~

\noindent\textbf{Initialization:}
This class implements \eqref{GP}, requires two mandatory arguments, \texttt{Prog}, \texttt{Rng} and two optional arguments \texttt{H} and 
\texttt{Settings}.
\begin{itemize}
	\item{\texttt{Prog=[f, Cons]} is a list consists of two parts:
	\begin{itemize}
		\item{`\texttt{f}' is the polynomial which is subject to minimizations.}
		\item{`\texttt{Cons=[g1, ..., gm]}' is a list of polynomials corresponding to the constraints $g_j\ge0$, for $j=1,\cdots,m$.}
	\end{itemize}
	They must be elements of the polynomial ring \texttt{Rng}, `\texttt{PolynomialRing(F,'x',n)}' where \texttt{n} is 
	the number of variables and `\texttt{F}' is the base field.}
	\item{\texttt{H} is the matrix $A=(a_{kl})$.}
	\item{\texttt{Settings} is a dictionary with the following keys:
	\begin{itemize}
		\item{`\texttt{Iterations}' is the number of iterations the \texttt{solver} tries to find an optimum solution. Its default value is 150.}
		\item{`\texttt{Details}' is by default set to \texttt{False}. Set \texttt{True} to see the detail output of the \textsc{CvxOpt} solver.}
		\item{`\texttt{tryKKT}' is by default $20$ that forces the solver to continue for the given number of steps, even if a singular KKT matrix 
		occurred.}
		\item{`\texttt{AutoMatrix}' accepts a Boolean `\texttt{True}' or `\texttt{False}'. The value \texttt{True} leaves the choice of $A$ to 
		the program automatically.}
	\end{itemize}
	}
\end{itemize}

\noindent\textbf{Running:}
The method \texttt{GPTools.minimize()} solves \eqref{GP} for \texttt{f} and \texttt{g1,...,gm}. 
The attribute \texttt{GPTools.fgp} also contains the output of \texttt{minimize}.

\noindent\textbf{Output:}
The attribute \texttt{GPTools.Info} is a dictionary which gives detailed information on the results of the method \texttt{minimize}.
\begin{itemize}
	\item{`\texttt{gp}' is the lower bound obtained from \eqref{GP}.}
	\item{`\texttt{Wall}' contains the wall time consumed by the solver.}
	\item{`\texttt{CPU}' contains the CPU time consumed by the solver.}
	\item{`\texttt{status}' and `\texttt{Message}' give some information about the running process.}
\end{itemize}

%%%%%%%%%%%%%%%%%%%%%%%%%%%%%%%%%%%%%
\subsection{semidefinite.SosTools(Prg, Rng [, Settings])}~

This class is on its early stage and aims to provides the same functionality as \textsc{SosTools} does for \textsc{Matlab}.

\noindent\textbf{Initialization:}
This class implements the semidefinite program \eqref{CSOS}.
\begin{itemize}
	\item{\texttt{Prg=[f,Cons]} is a list consists of two parts:
	\begin{itemize}
		\item{`\texttt{f}' is the polynomial which is subject to minimizations.}
		\item{`\texttt{Cons=[g1, ..., gm]}' is a list of polynomials corresponding to the constraints $g_j\ge0$, for $j=1,\cdots,m$.}
	\end{itemize}
	They must be elements of the polynomial ring \texttt{Rng}, `\texttt{PolynomialRing(F,'x',n)}' where \texttt{n} is 
	the number of variables and `\texttt{F}' is the base field.}
	\item{\texttt{Settings} is exactly similar to \texttt{GPTools} with two extra indices `\texttt{'Solver'}' and `\texttt{Order}' which corresponds to the index $k$ for $\textrm{Prg}_k$
	in \eqref{CSOS}.
	The default value for \texttt{Order} is set to be 0 which results in the minimum size for moment matrices.
    `\texttt{'Solver'}' assigns the sdp solver subject to the availability. Currently could be either `\texttt{cvxopt}' or `\texttt{csdp}'.}
\end{itemize}

\noindent\textbf{Running:}
\begin{itemize}
	\item{The method \texttt{SosTools.init\_sdp()} initializes the semidefinite program corresponding to \eqref{CSOS}.}
	\item{The method \texttt{SosTools.minimize()} returns the an optimum value $\rsos{f}$, for \eqref{CSOS} if exists.}
	\item{The method \texttt{SosTools.decompose} returns a vector of polynomials $[p_1,\dots,p_s]$ such that $f=\sum_{i=1}^s p_i^2$.}
\end{itemize}

\noindent\textbf{Output:}
The attribute \texttt{SosTools.Info} returns the similar information as for \texttt{GPTools.Info}. But instead of `\texttt{gp}' it has the key `\texttt{min}'
for $\rsos{f}$. The Info key `\texttt{is sos}' is by default \texttt{False}, but after executing `\texttt{decompose}' it may become \texttt{True}.

%%%%%%%%%%%%%%%%%%%%%%%%%%%%%%%%%%%%%
\subsection{Example}
\begin{verbatim}
from CvxAlgGeo import *
## Initial settings ##
n = 3			# Number of variables
d = 4			# Maximum degree of polynomials
m = 2			# Number of constraints
num_monos = 4	# Max number of monomials
R = PolynomialRing(RR,'x',n);
X = R.gens();

##########################################

Prg = []
Cns = []
f = R.random_element(d, randint(1, num_monos)) + sum(p^d for p in X)
Prg.append(f)
for i in range(m):
    g = R.random_element(d, num_monos)
    Cns.append(g)
Prg.append(Cns)

conf = {'Details':False,'tryKKT':20, 'AutoMatrix':False, 'Order':1}
M = identity_matrix(QQ, m+1)

print Prg

A = geometric.GPTools(Prg, R, H=M, Settings=conf)
A.minimize()
show(A.H)
print A.Info
B = semidefinite.SosTools(Prg, R, Settings=conf)
B.init_sdp()
B.minimize()
print B.Info
\end{verbatim}
The code generates random polynomials and computes $\rgp{f}$ and $f_{gp,\textbf{g}}$: Here is the output:
\begin{verbatim}
[x0^4 + x1^4 + x2^4 - x2^3 + 122*x0, 
[-x0^2*x1 - 4*x2^3 - x1, -x0^2*x1*x2 + 10*x1^3*x2 - 2*x1^2*x2]]

{'status': 'Optimal', 'Message': 'Optimal solution found by solver.', 
'Wall': 0.2654759883880615, 'CPU': 0.1799999999999926, 'gp':-285.988059619120}

{'status': 'Optimal', 'min': -285.98744579692226, 'Wall':1.1604969501495361, 
'Message': 'Feasible solution for moments of order 3', 
'CPU': 0.8499999999999943, 'Size': [20, 84]}
\end{verbatim}

%%%%%%%%%%%%%%%%%%%%%%%%%%%%%%%%%%%%%%%%%%%%%%%%%%%%%%%%%%%%%%%%%%%%%
\begin{thebibliography}{20}
	\bibitem{lbgpLasserre} M. Ghasemi, J. B. Lasserre and M. Marshall, \emph{Lower bounds on the global minimum of a polynomial}, to appear.
	\bibitem{lbgp} M. Ghasemi and M. Marshall, \emph{Lower bounds for polynomials using geometric programming}, SIAM J. Optim. 22(2) (460-473), 
	2012.
	\bibitem{genlbgp} M. Ghasemi and M. Marshall, \emph{Lower Bounds for a Polynomial on a basic closed semialgebraic set using geometric 
	programming}, in progress.
	\bibitem{gloptipoly} D. Henrion and J. B. Lasserre, \emph{Gloptipoly: global optimization over polynomials with
	matlab and sedumi}, ACM Transactions on Mathematical Software, 29(2):165–195, 2003.
	\bibitem{LasserreSDP} J. B. Lasserre, \emph{Global optimization with polynomials and the problem of moments}, SIAM J. Opt. 11(3)(796–817), 2001.
	\bibitem{LasserreMPP} J. B. Lasserre, \emph{Moments, Positive Polynomials and Their Applications}, Imperial College Press Optimization Series 
	Vol. 1, 2010.
	\bibitem{Laurent} M. Laurent, \emph{Sums of squares, moment matrices and optimization over polynomials}, 2010.
	\bibitem{sostools} S. Prajna, A. Papachristodoulou, P. Seiler, and P. A. Parrilo, \emph{SOSTOOLS: Sum of squares optimization toolbox for 
	MATLAB}, 2004
\end{thebibliography}
\end{document}